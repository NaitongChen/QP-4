\section{Conclusion and Future Directions}
\cite{meyer2022adjusting} proposes a Bayesian model for conducting seroprevalence analysis that accounts for test-kit performance. By applying this model to a dataset from a seroprevalence study conducted in New York state in early 2020 and comparing the results to the non-Bayesian version of the same analysis, we see that both analyses provide similar point estimates of regional cumulative incidences. However, the Bayesian credible intervals used to quantify uncertainties around each estimate, compared to their non-Bayesian counterparts, are narrower, more interpretable, and stay well-defined (non-negative).\\
\newline$ $
In this report, we extend the above Bayesian model by incorporating the approach in \cite{lewin2022seroprevalence} to additionally account for seroreversion. By comparing the newly proposed model to its non-Bayesian counterpart using seroprevalence data from Quebec, Canada, we see that the claims in \cite{meyer2022adjusting} largely hold. Namely, the point estimates from both models are similar, with the Bayesian credible intervals being narrower and more centred around the point estimates. In fact, we do come across instances of the non-Bayesian uncertainty intervals having a lowerbound below zero.\\
\newline$ $
One of the major motivations for approaching seroprevalence analysis in a Bayesian way is to improve the interpretability of the uncertainty intervals associated with each estimated cumulative incidence. In particular, with the way that uncertainty intervals are constructed in the non-Bayesian analysis in \cite{meyer2022adjusting}, they are not valid confidence intervals. Therefore, as a future direction of research, it is of interest to develop more theoretically justified ways to aggregate individual confidence intervals to produce uncertainty intervals that remain as valid confidence intervals. One idea is to use the Bonferroni correction to adjust the confidence level of each individual confidence interval so that the resulting interval has a desired confidence level. The degree Bonferroni correction here depends on the number of individual confidence intervals (or sources of uncertainty) that we use to produce the final interval. This is based on the observation that all cumulative incidence estimators discussed in this report are strictly monotone in each parameter of interest. However, this approach does not rule out the possibility of obtaining negative point estimates or negative uncertainty interval endpoints.\\
\newline$ $
Another direction for future research involves more formally testing the sensitivity of the proposed Bayesian models to the specification of prior distributions. As discussed before, the priors on cumulative incidences are based on cumulative reported case counts, and the priors on test sensitivity, test specificity, as well as seroreversion proportion are based on (external) validation studies. These sources of prior distributions contain variabilities within themselves, and so it is crucial to assess prior sensitivity beyond merely testing a number of plausible prior specifications. Should the model be sensitive to prior specification, more investigation may be conducted to help stablize the output of the corresponding Bayesian models.