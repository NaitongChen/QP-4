\section{Bayesian Seroprevalence Analysis Compared to its Frequentist Counterpart}
\subsection{Significance}
One of the most prominent advantages of the fully Bayesian approach to adjusting cumulative incidence for test-kit performance is that the quantified uncertainty is more interpretable compared to its non-Bayesian counterpart. This is precisely the motivation behind the development of this Bayesian procedure by \cite{meyer2022adjusting}. In the Bayesian version of the analysis, \cite{meyer2022adjusting} expresses the uncertainty of the cumulative incidence for a given region using a 95\% equal-tailed credible interval of the corresponding marginal posterior distribution. As an example, the credible interval of the cumulative incidence in the $i^\text{th}$ region in the $j^\text{th}$ super-region $s_{ij}$ can be constructed as follows. Following the notation from the previous section, we begin by writing out the marginal posterior density for $s_{ij}$:
\[
&p(s_{ij} \given X, N)\\
&\propto \int_{se}\int_{sp} p(se)p(sp) \left(\prod_{(r,q) \neq (i,j)} \int_{s_{rq}} p(s_{rq})p(x_{rq} \given n_{rq}, sp, se, s_{rq}) \partial s_{rq}\right) p(s_{ij})p(x_{ij} \given n_{ij}, sp, se, s_{ij}) \partial se \partial sp.
\]
Note that here the normalization constant is absored by proportionality. Given $p(s_{ij} \given X, N)$, the 95\% equal-tailed credible interval for $s_{ij}$ corresponds to the range of possible $s_{ij}$ values that covers the middle 95\% of the area under $p(s_{ij} \given X, N)$. In other words, if we were to sample from this marginal posterior distribution, there is a 95\% chance that the sampled value is within the credible interval. Furthermore, since the prior distribution on $s_{ij}$ ($p(s_{ij})$ in the expression of $p(s_{ij} \given X, N)$) is Beta and has support $[0,1]$, it is guaranteed by construction that the credible interval is contained in $[0,1]$.\\
\newline$ $
On the other hand, by following the non-Bayesian approach in \cite{meyer2022adjusting}, we may obtain negative estimates or uncertainty intervals that cross zero. This can be seen from the cumulative incidence correction equation in \cref{sec:frequentist}. When the sample proportion of positive tests $\hat{p}$ is smaller than $1-\hat{sp}$, the resulting estimate of the cumulative incidence would be negative. At the same time, note that the interval reprensenting the uncertainty around the estimated cumulative incidence is obtained by plugging in endpoints of two independent confidence intervals (one on test specificity and one on test sensitivity). We know that each 95\% confidence interval is a realization of all possible intervals that overall have a 95\% chance of covering what is regarded as the underlying true value. Then the proportion of pairs of 95\% confidence intervals on test specificity and test sensitivity covering the true values simultaneously is less than 95\%. As a result, by constructing an interval through aggregating two independent 95\% confidence intervals, the resulting interval would not be a valid 95\% confidence interval. This complicates the interpretation of the resulting interval. Furthermore, uncertainty intervals constructed this way do not take into account the uncertainties around the sample proportion of positive tests. While the other approach in \cite{rosenberg2020cumulative} takes into account both the uncertainties of the sample proportion of positive tests as well as test-kit performance, the use of three intervals still makes the quantified uncertainty not as easily interpreted as that under the Bayesian framework. Finally, we note that this approach may still result in negative estimated cumulative incidences.

\subsection{Limitations and challenges}
We begin by clarifying the notion of hierarchical priors. \cite{meyer2022adjusting} claims that the model they have constructed entails a hierarchical prior structure. However, this is not the case. The simplest Bayesian hierarchical models contain three layers: likelihood of the parameter of interest given data, a prior distribution that governs the parameter of interest, and a hyper-prior distribution that governs the prior distribution. However, the Bayesian model defined in \cite{meyer2022adjusting} does not contain any hyper-prior distributions that govens the prior distribution on cumulative incidence, test specificity, or test sensitivity. Therefore, while the Bayesian model constructed in \cite{meyer2022adjusting} does leverage prior information unique to each super-region, the use of the term hierarchical prior here is not precise.\\
\newline$ $
Following up on use of prior distributions that are unique to each super-region, the grouping of regions into super-regions remain somewhat arbitrary. For example, \cite{meyer2022adjusting} does not provide an argument for combining Westchester and Rockland Counties with Long Island into one super-region. It would be of interest to know how sensitive the Bayesian model is to the grouping of regions into super-regions. After the structure of the model is determined, we still need to consider the sensitivity of the model to prior specification. While \cite{meyer2022adjusting} tested a set of non-informative, weakly informative, and informative priors on the regional cumulative incidences, the same sensitivity analysis should ideally be carried out for test sensitivity and test specificity as well. This is because there are vast variabilities among estimated COVID-19 antibody test sensitivity and specificity from different studies. As an example, test sensitivity is estimated to be as low as $28$\% in \cite{noordin2022sero}. Should there be a clear difference in the resulting estimates, further investigations need be carried out in order to produce reliable estimates of cumulative incidences for each region.\\
\newline$ $
Finally, we note that both the Bayesian and non-Bayesian analyses considered here do not account for seroreversion, which is the loss of antibody detectability. As we obtain seroprevalence data from later stages of the pandemic, we run into higher risks of underestimating the proportion of population previously infected by COVID-19. This is because antibodies for the virus that causes COVID-19 may become undetectable after some period of time since infection. Individuals who have seroreverted would not count towards the sample proportions of positive antibody tests. However, it is important to account for these individuals in many cases. An example is when the goal is to estimate the case fatality rate of COVID-19, where it is crucial to include those who have seroreverted to avoid overestimating the case fatality rate \citep{brazeau2022estimating}. As a result, both the Bayesian and non-Bayesian analyses discussed above do not generalize to seroprevalence studies conducted later on in the pandemic as one might have hoped. In the following section, we propose a modification to the framework discussed above to account for seroreversion, and apply this modified Bayesian model to a serial seroprevalence study conducted in Quebec, Canada between 2020 and 2021.