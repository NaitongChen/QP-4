\section{Summary}
Over the past few years, tracking the spread of COVID-19 has been crucial to the development of a scientific understanding the disease, which ultimately guided public health policies aimed at controlling the spread of the disease across the world. As with any epidemic/pandemic, reported case counts wthin a defined geographic region is one of the most accessible statistics indicating the scale of the spread of a disease. However, due to testing availability, there may be many individuals in a given region that has been infected but not tested. This means that the total number of infection may be much higher than those reflected from reported case counts (cite). \\
\newline$ $
To more accuratly estimate the cumulative number of infections over a period of time, an alternative approach to looking at the reported case counts is to conduct population-based seroprevalence studies. 