\section{Summary}
Over the past few years, tracking the spread of COVID-19 has been crucial to developing a scientific understanding the disease, which ultimately guided public health protocols aimed at controlling the spread of the disease across the world. As with any epidemic/pandemic, reported case counts wthin a defined geographic region is one of the most accessible statistics indicating the scale of the spread of a disease. However, due to testing availability, there may be many individuals in a given region that have been infected but not tested. This means that the total number of infection may be much greater than that reflected from reported case counts (cite).\\
\newline$ $
To more accuratly estimate the cumulative number of infections over a period of time, an alternative approach is to conduct population-based seroprevalence studies. To carry out a seroprevalence study for a given region on a given disease, researchers begin by obtaining a sample representative of the population. Antibody tests are then performed for the disease of interest over each individual in the sample. A positive antibody test indicates a case of infection of the tested disease. Therefore, the proportion of positive tests in the sample can be used as an estimate of the porportion of population infected with the disease over some time interval, which we call the cumulative incidence. With the estimated cumulative incidence, one can estimate the total number of infected individuals in the population. It is worth noting, however, that seroprevalence studies cannot identify previous infections whose antibodies are no longer detectable or recent infections that have yet to produce detectable antibodies. At the same time, they also do not include individuals that have died after becoming infected. As a result, a seroprevalence study as we describe it here is only informative about cumulative incidence for the average period over which antibodies are detectable, provided that the disease has a relatively low fatality rate.\\
\newline$ $
We know that COVID-19 has a relatively low fatality rate (cite). We also know that an individual starts to produce detectable antibodies after an average of 25 days since infection, and that the antibodies stay detectable for months after infection (cite). Therefore, using data from a seroprevalence study conducted within the first few months of the pandemic, we can estimate cumulative incidence over the period from the beginning of the pandemic until roughly a month prior to when the samples were taken.\\
\newline$ $
Since antibody tests are not 100\% accurate, there may be positive cases that test negative and negative cases that test positive. Therefore, one would ideally also like to adjust cumulative incidence for test-kit performance. This is typically done as follows. We begin by defining test specificity $sp$ as the proportion of noncases that test negative and test sensitivity $se$ as the proportion of actual cases that test positive. Then with the true cumulative incidence being denoted as $s$, we can model the observed prevalence (proportion of positive tests in the sample) of a given region $p$ as
\[
p = s \times se + (1-s) \times (1 - sp).
\]
To put in words, the observed prevalence can be decomposed into proportion of actual cases that correctly test positive and noncases that incorrectly test positive. Given the total sample size $n$ from a given region and the number of positive tests $x$ from the sample, it is reasonable to assume that
\begin{align}
x \given n, p \distas \distBinom(n, p).
\end{align}
Then we can construct a bayesian model by defining a set of prior distributions on each of $s, se$, and $sp$ using distributions with a support on $[0,1]$ and viewing the above as the likelihood. (cite) applies this bayesian model to a dataset obtained from a seroprevalence study conducted in New York state between April 19 and April 28 in 2020. This dataset contains the number of positive antibody tests and the total number of tests from each of 11 regions across New York state. Full details of data can be found in (citation). Adjusting for the average time between infection and when antibodies become detectable, this dataset can be used to estimate cumulative incidences from the beginning of the pandemic until Mar 29, 2020. This is because there are 25 days between Mar 29, 2020 and the seroprevalence study midpoint April 23, 2020.\\
\newline$ $
Instead of using the same prior for cumulative incidence for each region, Meyer notes it is possible that regions close to each other may share similar sociodemographic factors which may be associated to number of infections. As a result, they created three super-regions that share similar proportions of positive tests and used these super-regions to define a hierarchical model as follows. 
\[
model.
\]
Priors based on reported cases, sensitivity and specificity based on validation studies (show CI here). Can run MCMC to get samples for each region, then aggregate through weights according to population. Use median as point esimate and 95 credible interval as a measure of uncertainty.\\
\newline$ $
Meyer compared this to a non-Bayesian version of the same analysis. Use sample proportion with mean sensivity and specificity as point estimate. Quantify uncertainty using CI end points. (rederive equation). Overall point estimate similar, but uncertainty interval narrower and no negative values. We elaborate on pros and cons in the following sections.