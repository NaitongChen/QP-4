\section{Summary}
Over the past few years, tracking the spread of COVID-19 has been crucial to developing of a scientific understanding the disease, which ultimately guided public health policies aimed at controlling the spread of the disease across the world. As with any epidemic/pandemic, reported case counts wthin a defined geographic region is one of the most accessible statistics indicating the scale of the spread of a disease. However, due to testing availability, there may be many individuals in a given region that have been infected but not tested. This means that the total number of infection may be much higher than that reflected from reported case counts (cite).\\
\newline$ $
To more accuratly estimate the cumulative number of infections over a period of time, an alternative approach is to conduct population-based seroprevalence studies. To carry out a seroprevalence study for a given region on a given disease, researchers begin by obtaining a sample representative of the population. Antibody tests are then performed for the disease of interest over each individual in the sample. A positive antibody test indicates a previous infection of the disease of interest. Therefore, the proportion of positive tests in the sample can be used as an estimate of the porportion of population previously infected with the disease, which we call the cumulative incidence. With the estimated cumulative incidence, one can estimate the total number of all individuals in the population with a previous infection. It is worth noting, however, that seroprevalence studies cannot identify previous infections whose antibodies are no longer detectable or recent infections that have yet to produce detectable antibodies. At the same time, they also do not include individuals that have died after becoming infected. As a result, a seroprevalence study as we describe it here is only informative about cumulative incidence for the average period over which antibodies are detectable, provided that the disease has a relatively low fatality rate.\\
\newline$ $
We know that COVID-19 has a relatively low fatality rate (cite). We also know that an individual starts to produce detectable antibodies after an average of 25 days since infection, and that the antibodies stay detectable for months after infection (cite). Therefore, cumulative incidences estimated by seroprevalence studies conducted within the first few months of the pandemic can be relatively accurate over the period from the beginning of the pandemic until roughly a month prior to when the samples were taken.\\
\newline$ $
